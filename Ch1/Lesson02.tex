\section{Unit II: Fundamentals in Java}
\subsection{Class Logistics}
\begin{itemize}
	\item \underline{Introduction to the course}: Definition of the first program and Java \texttt{class}.
	Show what makes a computer program. Discuss what is needed (e.g., syntax, punctuation, semantics, keywords,
	reserved words). Explain the header of any program in Java (including the \texttt{main}). 
	\item \underline{Definition of variables}: introduce the eight primitive types of variables (\texttt{int},
	\texttt{short}, \texttt{long}, \texttt{double}, \texttt{float}, \texttt{byte}, \texttt{char}, and
	\texttt{byte}).  
	\item \underline{Definition of syntax to declare a variable}: Show how to declare/define a variable. 
	Show the syntax based on the types of variables and the values. Show how to initialize a variable. 
	How to declare and initialize at the same time. 
	\item \underline{Definition of \emph{scope} and \emph{values}}. Show the meaning of the program scope
	by introducing the \{\} (curly brackets) in the program. Show inner \emph{scopes} that will show visibility 
	of the variables and their corresponding values.
	
	\item \underline{How to manipulate a \texttt{String}}: Show students the type of \texttt{String}, as a collection
	of characters. A \texttt{String} is an object, however, the concept of object, methods, and constructors it is 
	introduced in Lesson 12. At this point show students the basic idea of creating a \texttt{String} and using 
	methods such as \texttt{toUpperCase}, \texttt{toLowerCase}, \texttt{length}, and \texttt{charAt}.
	
	\item \underline{Explain order of operations}: depending on the order of operation established in the program,
	Java will interpreted the order based on PEMDAS-order. Show students the different combinations that will
	result in different outcomes based on the order of operations in some arithmetic expressions. Also show the 
	division with \texttt{int}s and with \texttt{double}s, these operations will result in two different outcomes.
	
	\item \underline{Show the interactions between the user and Dr. Java}: Use Dr. Java's interactions window
	to explain the order of operations. 
	 
	\item \underline{Discuss the usage \texttt{Scanner}}: Explain why we need \texttt{Scanner} to interact with 
	the user and provide input/output. Explain methods such has \texttt{next()}, \texttt{nextLine()}, \texttt{nextInt()},
	and \texttt{nextDouble()}.
	 
	\item \underline{Discuss the usage \texttt{JOptionPane}}: Explain an alternative to interact with the user. Explain
	the trade-offs between \texttt{JOptionPane} and \texttt{Scanner}. Introduce the parsing component e.g., 
	\texttt{Integer.parseInt} to transform a \texttt{String} to an \texttt{int}.
	
	\item \underline{Introduce \texttt{Math} library}: Introduce the Math library with the Java API. show examples by using
	several mathematical operations e.g., \texttt{sin}, \texttt{abs}, \texttt{sqrt}, \texttt{cbrt}, and \texttt{pow}.
\end{itemize}

\subsection{SLOs Accomplished}
\begin{itemize}
	\item B.1.	understand the syntax of how to define and initialize a variable of different types
	\item B.2.	know the primitive types in java, and the difference between them
	\item B.3.	describe what a scope is and the visibility of a variable
	\item B.4.	identify the correspondence between the class and the java source code file name
	\item B.5.	define what a \texttt{String} is and the difference between \texttt{String} and other primitive types
	\item B.6.	explore the Java API for the \texttt{Math}, \texttt{String}, and \texttt{Scanner} library for usage of methods
	\item B.7.	write programs that interact with the user using \texttt{Scanner} and \texttt{JOptionPane}
	\item B.8.	extract information from the user to handle primitive types or Strings
\end{itemize}

\subsection{Observations}
\begin{itemize}
	\item Students are exposed for the first time to create a program from scratch, they need time to practice 
	the first program, i.e., define a java class and the main
	\item The equal sign (=) has a different meaning from the math context. In java programming is used as an assignation operator
	\item Students seems to be confused with the syntax, including upper/lower case. Explain that java is case sensitive
	\item Introduce the idea of \emph{objects} without going into the details. At this point the student know 3 objects (i.e.,
	the \texttt{String}, \texttt{System}, and the name of the class \texttt{MyFirstProgram}). Explain that we will introduce 
	this concept of objects in the 6th week of classes
	\item Show the difference between primitive types and \texttt{String} object.
	\item Student seems to confuse the definition of a \texttt{char} and \texttt{String}. Since a \texttt{String}
	is a collection of \texttt{char}s, students confuse the initialization of the values
	\item Explain the methods that a \texttt{String} has. The notion of \emph{methods} is premature at this point,
	since it is not known until Chapter 5. The word \emph{functions} seems to help
	\item The \texttt{String}'s method \texttt{charAt} extracts the character at certain position, the confusion 
	is that in Java starts counting at 0 and not 1. Students confuse this counting convention
	\item The students at this point know \texttt{System}, \texttt{Scanner}, \texttt{String}, \texttt{JOptionPane}, 
	\texttt{Math}, \texttt{Integer}, \texttt{Double}. Students start to relate that objects (with capital letter), 
	they use \emph{methods} by using the dot (``.''). 
\end{itemize}
\newpage
\subsection{Materials and/or Instruments to Illustrate Lecture's Content}
	\subsubsection{\texttt{MyFirstProgram.java}}
	\lstinputlisting[language=java]{Lesson01/MyFirstProgram.java}

	\subsubsection {\texttt{MySringManipulation.java}}
	\lstinputlisting[language=java]{lesson02/MySringManipulation.java}
	\newpage
	
	\subsubsection {\texttt{Interaction1.java}}
	\lstinputlisting[language=java]{lesson02/Interaction1.java}
	\newpage
	
	\subsubsection {\texttt{Interactions Example}}
	\begin{itemize}
		\item We can use basic operations
		\begin{verbatim}
			> 3+2
			5
			> 5-2
			3
		\end{verbatim}
		\item or we can declare different variables that will hold different values
		\begin{verbatim}
		> int x = 5
		> x
		5
		\end{verbatim}
		\item make sure the variable is declared first, otherwise will give you an error
		\begin{verbatim}
		> y
		Static Error: Undefined name 'y'
		// even if you provide a value, you need the "int" in front of the variable name
		> y = 2
		Static Error: Undefined name 'y'
		> int y = 2
		> y
		2
		> x
		5
		\end{verbatim}
		\item we can perform operations between variables, since they have values
		\begin{verbatim}
		> x+y
		7
		\end{verbatim}
		\item you can even declare a placeholder to store the arithmetic operations of 2 variables
		\begin{verbatim}
		> int z = x + y
		> z
		7
		\end{verbatim}
		\item you can use double as well as int or other primitive types
		\begin{verbatim}
		> double a = 2.1
		> a
		2.1
		> double b = 3.2
		> b
		3.2
		\end{verbatim}
		\item however, always be consistent with the declaration and assignation values you provide
		\begin{verbatim}
		> double c = a + b
		> c
		5.300000000000001
		> a+b
		5.300000000000001
		\end{verbatim}
		\item c is a double, but x and y are ints. If you add them and decide to store them into a double, java wil provide the decimal part for you
		\begin{verbatim}
		> c = x + y
		7.0
		\end{verbatim}
		\item however, the other way around is not legal since an int is a 1 byte and a double is a 4 bytes
		\begin{verbatim}
		> z = a + b
		Static Error: Bad types in assignment: from double to int
		> z = (a + b)
		Static Error: Bad types in assignment: from double to int
		\end{verbatim}
		\item in case you really want the whole number of a double you can "type cast" the result, i.e., force java to provide the value for the entire value
		\begin{verbatim}
		> z = (int)(a + b)
		5
		\end{verbatim}
		\item the division of two integers will give you an int, that's why we are missing the 0.5
		\begin{verbatim}
		> 5/2
		2
		\end{verbatim}
		\item the modulus (%) of two numbers will return the reminder of the division
		\begin{verbatim}
		> 5\%2
		1
		\end{verbatim}
		\item in addition to the regular arithmetic operations, you can use the Math library methods
		 see: http://docs.oracle.com/javase/7/docs/api/java/lang/Math.html
		\begin{verbatim}
		> Math.pow(2,3)
		8.0
		> Math.pow(2,10)
		1024.0
		> Math.sqrt(16)
		4.0
		> Math.PI
		3.141592653589793
		> Math.E
		2.718281828459045
		> Math.PI
		3.141592653589793
		> Math.E
		2.718281828459045
		> Math.sin(90)
		0.8939966636005579
		> Math.cos(90)
		-0.4480736161291702
		> Math.tan(90)
		-1.995200412208242
		\end{verbatim}
\end{itemize}
% \end{document}